% Chapter 1

\chapter{Thesis Contribution and Future Work} % Write in your own chapter title
\label{Chapter1}
\lhead{Chapter 1. \emph{Introduction}} % Write in your own chapter title to set the page header

\section{Introduction}
As we have discussed all the parameters in this thesis such as introduction, methodology, implementation, results etc. After results and experiments, this proposed system motivates us to do some more work on it. So, in this chapter, we will discuss the work that we will do in future. But before discuss the future work, we will discuss about the thesis contribution.

\subsection{Thesis Contribution}
 According to previous research papers, indoor localization based applications using different techniques. Most of the applications based on Wi-Fi. Image based indoor localization, capacitive sensors, Zigbee Wi-Fi are previous technique for indoor localization. In our research work, we have to make an android application which tells the indoor location of the user using BLE beacons. There are some points which describe that what our contribution in thesis which is as follows:

\begin{itemize}
\item Bluetooth beacons are much more compatible than other devices such as Wi-Fi, Image based indoor localization, capacitive sensors, Zigbee Wi-Fi.
\item BLE beacon consumes less energy. It is light weight and cheaper than Wi-Fi signals. BLE beacons are usually battery powered, which are more flexible. It is easier to sense the signals. 
\item Our system is much more compatible than other indoor based localization application.
\item Our system provides much more reliability, maintainability, security and safety, reusability than other indoor based localization applications.
\item It gives better performance than other indoor based localization applications.
\item Our System is user friendly and easy to use.
\item Our system well-defined the location of the user than other indoor based localization applications.
\item Our system also tells the entire information of the room in which user is present which other indoor based localization applications not tell.
\item It also tells the entire information of nearby room which other indoor based localization applications not tell.
\item Our system tells the information of the rooms and nearby rooms in the form of texture, pictures and audio format.
\item Our proposed system supports the mobile devices such as  tablets and smart phones
\item Our proposed system supports the Android OS (Operating System).

\end{itemize}

\subsection{Future Work}
To improve the accuracy and performance of this purposed system, we want to do further work on them. We also want to do this project on commercial level. Actually the idea of our project is world changing. But due to limited resources, we implement this idea on a specific workspace. Here we discuss some points in which we will implement in the future.\

\begin{itemize}
\item First of all, we want to develop an algorithm which gives more precise accuracy than previous algorithms such as k-NN, ANN and RF (Random Forest). 
\item We want to work on a big data set such as data set of all over university. For this purpose we will capture the data of all over the university.
\item We will develop an application with different database. Actually, the database of one department of the university includes rooms, room members and office hours of the room members. But when we will work on the university, the database is different.
\item For our proposed system, we use online database. But in future, we want to make an offline database. So that, when users install the app in the mobile, the overall data is also installed in the mobile.
\item We will develop an android application for our CSE department, UET Lahore. But in future we will make this application for whole university.
\item If we will have more resources in the future, we will make this system for other places such as hospitals, offices, shopping malls and other universities
\end{itemize}.



